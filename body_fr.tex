\begin{minipage}{0.75\linewidth}
	\Huge{Thomas Duboucher} \\
	\supweb{\Large{www.duboucher.eu}}{http://www.duboucher.eu/} \\
	\\
	\hspace{2.6cm}\huge{Jeune diplômé Supélec}\\
	\hspace{2.2cm}\LARGE{\textit{Ingénieur Sécurité des Systèmes}}
\end{minipage}
\begin{minipage}{0.25\linewidth}
	\supaddress{13 rue Victor Hugo\\ 92130 Issy-les-Moulineaux} \\
	\supmobile{06 43 10 34 40} \\
	%\supphone{01 47 52 05 93} \\
	\supmail{thomas@duboucher.eu}
\end{minipage}

%\supspacer

\begin{supitemize}{Formation}
	\supitem[*]{\supdate*{2011}}{\supbold{Mastère en Recherche Informatique}, \supemph{université de Rennes-1}}
		{Parcours Sécurité des Contenus et des Infrastructures Informatiques}
	\supitem[*]{\supdate{2007}{2011}}{\supbold{\'Ecole Supérieure d'\'Electricité}}
		{Majeure Systèmes d'Information Sécurisés}
	\supitem[*]{\supdate{2005}{2007}}{\supbold{Classes Préparatoires}, \supemph{filière TSI}}
		{}
	\supitem[*]{\supdate*{2005}}{\supbold{Baccalauréat}, \supemph{filière STI Série \'Electronique}} % (Sciences et Techniques Industrielles)
		{Obtenu avec la Mention Très Bien}
\end{supitemize}

\begin{supitemize}{Expériences}
	\supitem[*]{\supdate{avril 2011}{août 2011}}{\supbold{Stage Ingénieur \& Recherche}, \supemph{Ministère de la Défense}}
		{}
	\supitem[*]{\supdate{2010}{2011}}{\supbold{Administrateur Système}, \supemph{FedeRez}}
		{
			Administration de serveurs sous GNU/Linux
			\begin{itemize}
				\item Conférencier lors des Journées FedeRez 2011
			\end{itemize}
		}
	\supitem[*]{\supdate*{\'Eté 2010}}{\supbold{Stage Technique}, \supemph{Thales Communications}, \supemph{2 mois}}
		{}
		%{Implémentation d'algorithmes d'ordonnancements Temps-Réel sous VxWorks}
	\supitem[*]{\supdate*{\'Eté 2009}}{\supbold{Stage Technique}, \supemph{EADS D\&S}, \supemph{2 mois}}
		{}
		%{Implémentation d'un chargeur d'amorçage avec module TPM}
	\supitem[*]{\supdate{2007}{2009}}{\supbold{Trésorier d'un Club de Robotique}, \supemph{Sumo Supélec}, \supemph{Gif-sur-Yvette}}
		{Participation pour deux saisons à la Coupe de France de Robotique}
	\supitem[*]{\supdate{2007}{2009}}{\supbold{Administrateur Réseau}, \supemph{Supélec Rézo}, \supemph{Gif-sur-Yvette}}
		{
			Administration du réseau informatique de la résidence de Supélec, Gif-sur-Yvette
			\begin{itemize}
				\item Déploiement de la diffusion de la TNT sur réseau IP
				\item Mise en place d'un réseau VPN multisites
			\end{itemize}
		}
	\supitem[*]{\supdate{2007}{2009}}{\supbold{Vice-Président du club GNU/Linux}, \supemph{Supélec}, \supemph{Gif-sur-Yvette}}
		{Organisation de séances de découverte et d'initiation aux Logiciels Libres}
\end{supitemize}

\begin{supitemize}{Langues}
	\supitem{Anglais}{Niveau B2 - Avancé}
		{
			Plusieurs séjours dans différents pays étrangers, dont \'Egypte, Jordanie et Turquie
			\begin{itemize}
				\item Séjour linguistique de 2 semaines à Brighton, Angleterre
				\item Séjour linguistique de 6 semaines à Santa Rosa, Californie
			\end{itemize}
		}
	\supitem{Espagnol}{Notions}
		{}
	\supitem{Chinois}{Notions}
		{Stage d'exécution et linguistique de six semaines effectué à Shanghai, Chine}
\end{supitemize}

\begin{supitemize}{Compétences}
	\supitem{Langages de Programmation}{}
		{
			Utilisation courante de C, C++, Java, Lua, scripts shell\newline
			Connaissances en Perl, Python, Ruby, SQL, VHDL, Assembleur \emph{x86} et \emph{ARM}
		}
	\supitem{Logiciels}{}
		{Eclipse, Visual Studio, Matlab \& Simulink, Quartus, \LaTeX}
	\supitem{Systèmes d'exploitation}{}
		{VxWorks, L4, Android et GNU/Linux (Ubuntu, Fedora et Arch Linux)}
	\supitem{Réseau}{}
		{TCP/IP, multicast, VPN, équipements professionnels (HP, Cisco, 3Com)}
	\supitem{Sécurité}{}
		{Cryptographie, SSL/TLS, IPsec, PenTest, analyse statique, IDS, politiques de sécurité}
\end{supitemize}

\begin{supitemize}{Centres d'Intérêts}
	\supitem{Logiciel Libre}{}
		{Contributions à plusieurs projets}
	\supitem{Sécurité des Systèmes d'Information}{}
		{Virtualisation, Architectures \ensuremath{\mu}-kernel, Informatique de Confiance}
	\supitem{Systèmes Embarqués}{}
		{Systèmes Temps-réel, Architectures Reconfigurables}
%	\supitem{Domotique}{}
%		{Réseaux Domestiques, Interfaces Homme-Machine}
\end{supitemize}
